\documentclass{article}

\usepackage[utf8]{inputenc}
\usepackage[left=1.25in,top=1.25in,right=1.25in,bottom=1.25in,head=1.25in]{geometry}
\usepackage{amsmath,amssymb,amsthm}

\usepackage{import}
\usepackage{xifthen}
\usepackage{pdfpages}
\usepackage{transparent}
\usepackage{tcolorbox}
\usepackage{algpseudocode}

\newcommand{\incfig}[1]{%
    \def\svgwidth{\columnwidth}
    \import{./figures/}{#1.pdf_tex}
}
\usepackage{tikz}
\usetikzlibrary{shapes ,arrows ,positioning}
\newcommand{\floor}[1]{\lfloor #1 \rfloor}

\newtheorem{exo}{Exercise}
\def\P{\mathbb{P}}
\def\E{\mathbb{E}}

\title{Randomised Algorithms \\
Winter term 2022/2023, Exercise Sheet No. 9}

\author{
    \textbf{Authors:} \\
    Ben Ayad, Mohamed Ayoub \\
    Kamzon, Noureddine
}

\begin{document}
\maketitle


\begin{exo}{\ \\}

\noindent
\textbf{(a)} Let's say w.l.o.g  that $a^\star$ satifies the first literal (and only the first) of every clause, then $\neg{a}$ would satify the second literal of each clause, and hence, $\neg{a}$  would also be a satisfied assigment.


    
\end{exo}
\begin{exo}{\ \\}
\noindent 
\textbf{(a)} The new recurrence relations could be summed up as follows:
\[
\begin{cases}
    E_0 
    &= 0 \\
    E_i 
    &= 1 + \frac{1}{2} (E_{i+1}+E_{i-1})
    \quad \text{for: $i \in \{1, \dots, n-1\}$ }
    \\
    E_n 
    &= 1 + \frac{1}{2} (E_{n}+E_{n-1})\\
\end{cases}
\]
The last implies that: $E_n = 2 +E_{n-1} \implies D_n = 2$.\\   
The second line implies that:
\[
D_k = 2 + D_{k-1} \text{ for } k \in \{1, \dots, n-1\}
\]

By summing each equation from $k=1$ up to $k = i \leq n-1$, we get:
\[
D_1 = 2k + D_{k+1}
\]

Particularly when $k=n-1$
\[
D_1 = 2(n-1) + D_{n} = 2n = E_1
\]

And hence, $D_{k+1} = 2(n-k)$, for $k \in \{1, \dots, n-1\}$
   
\noindent 
\textbf{(b)} For $k \in \{1, \dots, n\}$, we have:
\begin{align*}
E_k 
&= \sum^{k}_{i=1} D_i  \\   
&= \sum^{k}_{i=1} 2(n-(k-1)) \\
&= k (2n -k) + k
\end{align*}


\end{exo}

\begin{exo}{\ \\}
\noindent
\textbf{(a)} The figure below represents the Markov Chain of interest:  
    
\begin {figure}[h]
    \centering 
    \begin{tikzpicture}[ ->  , shorten>=2pt , line width =0.5pt , node distance =2cm]
        \node [circle , draw ] (zero) {$\emptyset$};
        \node [circle , draw ] (one) [ right of = zero ] {$T_1$};
        \node [circle , draw ] (two) [ right of = one ] {$T_2$};
        \node [circle , draw ] (three) [ right of = two ] {$H_1$};
        \node [circle , draw ] (foor) [ right of = three ] {$T_3$};
        \path (zero) edge [loop left]  node {$.5$} (zero) ;
        \path (zero) edge [bend left] node [above] {$.5$} (one) ;
        \path (one)  edge [bend left] node [above] {$.5$} (zero) ;
        \path (one) edge [bend left] node [above]{$.5$} (two) ;
        \path (two) edge node [ above ]{$.5$} (three) ;
        \path (two) edge [ bend left ] node [below]{$.5$} (one) ;
        \path (three) edge [ bend left] node [below]{$.5$} (zero) ; 
        \path (three) edge [below] node {$.5$} (foor) ; 
        \path (foor) edge [loop right] node {$0.5$} (foor) ;
    \end{tikzpicture}
\end{figure}



\noindent
\textbf{(b)} 

\end{exo}


\end{document}
