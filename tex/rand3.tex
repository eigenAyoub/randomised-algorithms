\documentclass{article}

\usepackage[utf8]{inputenc}
\usepackage[left=1.25in,top=1.25in,right=1.25in,bottom=1.25in,head=1.25in]{geometry}
\usepackage{amsmath,amssymb,amsthm}

\usepackage{import}
\usepackage{xifthen}
\usepackage{pdfpages}
\usepackage{transparent}

\newcommand{\incfig}[1]{%
    \def\svgwidth{\columnwidth}
    \import{./figures/}{#1.pdf_tex}
}



\newtheorem{exo}{Exercise}
\def\P{\mathbb{P}}
\def\E{\mathbb{E}}

\title{Randomised Algorithms \\
Winter term 2022/2023, Exercise Sheet No. 3}

\author{
    \textbf{Authors:} \\
    Ben Ayad, Mohamed Ayoub \\
    Kamzon, Noureddine
}

\begin{document}

\maketitle


\begin{exo}{\ \\}

\end{exo}


\begin{exo}{\ \\}

\end{exo}


\newpage
\begin{exo}{\ \\}

\textbf{(a)}
We start from Top to Bottom, we assign 1 to the root, and follow these two startegies  to assign the levels below until we reach the leaves:\\

If the parent is $\lor$:
\begin{itemize}
    \item First child: $0$
    \item Second child: Parent Value
\end{itemize}

If the parent is $\land$:
\begin{itemize}
    \item First child: $1$
    \item Second child: Parent Value
\end{itemize}

\textbf{(b)} The following figures captures the algorithm:
 
\begin{figure}[ht]
    \centering
    \incfig{graph-incscape}
    \caption{graph-incscape}
    \label{fig:graph-incscape}
\end{figure}


\end{exo}

\begin{exo}{\ \\}

\end{exo}



\end{document}

