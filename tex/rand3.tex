\documentclass{article}

\usepackage[utf8]{inputenc}
\usepackage[left=1.25in,top=1.25in,right=1.25in,bottom=1.25in,head=1.25in]{geometry}
\usepackage{amsmath,amssymb,amsthm}

\usepackage{import}
\usepackage{xifthen}
\usepackage{pdfpages}
\usepackage{transparent}

\newcommand{\incfig}[1]{%
    \def\svgwidth{\columnwidth}
    \import{./figures/}{#1.pdf_tex}
}



\newtheorem{exo}{Exercise}
\def\P{\mathbb{P}}
\def\E{\mathbb{E}}

\title{Randomised Algorithms \\
Winter term 2022/2023, Exercise Sheet No. 3}

\author{
    \textbf{Authors:} \\
    Ben Ayad, Mohamed Ayoub \\
    Kamzon, Noureddine
}

\begin{document}

\maketitle


\begin{exo}{\ \\}

\noindent
\textbf{(a)} Comparisons between 15 and 8:
\begin{itemize}
    \item Pivot is 7: Decision will be postponed.
    \item Pivot is 15: They will be compared.
    \item Pivot is 10: They will be imediately seperated and hence never be compared.
\end{itemize}
   
\noindent
\textbf{(b)} The probability of 8 and 15 being compared, is the probability that the first pivot to be selected from the set \{Numbers in the input between 8 and 15\} is either 8 or 15:

$\frac{2}{|\{\text{Numbers in the input between $8$ and $15$}\}|} = \frac{2}{|\{8,11,19,10,15\}|} = \frac{2}{5}  $


\end{exo}


\begin{exo}{\ \\}

We assume for questions (a) and (b) that $\frac{1}{2}+\frac{1}{\sqrt{n}} < 1$, which yields that $n > 4$.   \\
\noindent
\textbf{(a)}
Based on the lecture notes, for a letter $x$ of size $n$, we would have: $ \P[\{ \text{{$A_t$ failing}} \}] \leq (1-\frac{4}{n})^{\frac{t}{2}} $ \\


\noindent
\textbf{(b)} Let's say we run $\mathcal{A}$, $2n^k$ times, for a positive integer $k$. The probablity of $\mathcal{A}_{2n^k}$ failing is:


\begin{align*}
    \P[\text{\{$\mathcal{A}_{2n^k}$ failing\}}] &\leq (1-\frac{4}{n})^{n^k} \\
    &\leq e^{-(1-\frac{4}{n})n^k}   \text{\quad \quad (using the hint)} \\
    &= e^{4n^{k-1} - n^k} \\
    &\leq e^{-2n^k}  \text{\quad \quad (since $4n^{k-1} \leq n^k$)} \\
\end{align*}
 
For $k$ sufficiently large, the RHS is guarenteed to converge to 0, and hence $\forall \epsilon$ there exists $k$ s.t: the RHS is less than $1/2 - \epsilon$ \\

For $1/2 >\epsilon > 0$

\begin{align*}
    & e^{-2n^k} \leq \frac{1}{2} - \epsilon \\
    &\implies -2n^k \leq \ln(1/2 - \epsilon) \\
    &\implies n^k \geq -\ln(1/2 - \epsilon)/2 \\
    &\implies k \ln(n) \geq  \ln(-\ln(1/2 - \epsilon)/2) = \alpha \\
    &\implies k \geq \frac{\alpha}{\ln(n)} 
\end{align*}

Therefore, any positive $k$ verifying the past inequality would work. As $\mathcal{A}_{2n^k}$ still runs in polynomial time ($2n^k T(n)$), $L \in BPP$  \\ 


  \noindent
\textbf{(c)}
The probability of $\mathcal{A}$ failing for an input of size $n$ is the same of the probability of failing in the first case (when the probability is at least $\frac{1}{2}+\frac{1}{\sqrt{n}})$ for an input size of $n^2$. Hence, using \textbf{(b)}, $L \in BPP$.

\end{exo}


\newpage
\begin{exo}{\ \\}

\textbf{(a)}
We start from Top to Bottom, we assign 1 to the root, and follow these two startegies  to assign the levels below until we reach the leaves:\\

If the parent is $\lor$:
\begin{itemize}
    \item First child: $0$
    \item Second child: Parent Value
\end{itemize}

If the parent is $\land$:
\begin{itemize}
    \item First child: $1$
    \item Second child: Parent Value
\end{itemize}

\textbf{(b)} The following figures captures the algorithm:
 
\begin{figure}[ht]
    \centering
    \incfig{graph-incscape}
    \caption{graph-incscape}
    \label{fig:graph-incscape}
\end{figure}


\end{exo}

\begin{exo}{\ \\}


We have for $i,j \in \{1, \dots, n\}\colon$ \[ \min_{i} M_{i,j} \leq M_{i,j} \le \max_{j} M_{i,j}   \]
\end{exo}

Hence, for $i \in \{1, \dots, n\}$, and as the RHS is independent of $j\colon$ \[ \max_{j} \min_{i} M_{i,j} \le \max_{j} M_{i,j} \]

Notice that the LHS is a constant (independent of both $i$ and $j$), and the past inequality is verified $\forall i$. We finally get: \[ \max_{j} \min_{i} M_{i,j} \le  \min_{i} \max_{j} M_{i,j}  \]


\end{document}

