\documentclass{article}

\usepackage[utf8]{inputenc}
\usepackage[left=1.25in,top=1.25in,right=1.25in,bottom=1.25in,head=1.25in]{geometry}
\usepackage{amsmath,amssymb,amsthm}

\usepackage{import}
\usepackage{xifthen}
\usepackage{pdfpages}
\usepackage{transparent}

\newcommand{\incfig}[1]{%
    \def\svgwidth{\columnwidth}
    \import{./figures/}{#1.pdf_tex}
}



\newtheorem{exo}{Exercise}
\def\P{\mathbb{P}}
\def\E{\mathbb{E}}

\title{Randomised Algorithms \\
Winter term 2022/2023, Exercise Sheet No. 7}

\author{
    \textbf{Authors:} \\
    Ben Ayad, Mohamed Ayoub \\
    Kamzon, Noureddine
}

\begin{document}

\maketitle

\begin{exo}{\ \\}
\noindent
\textbf{(a)} 

We have for $t \geq 1$:  \quad \quad  $
X_{t+1}  = 
\begin{cases}
    X_t + [X_t/2]  \quad \quad &\text{with probability 18/37} \\
    X_t - [X_t/2]  &\text{Otherwise}
\end{cases}
$

Technically speaking, we can never reach the state $0$, as the player gets stuck when $X_t = 1$. We'll define $Y_t = X_t - 1$. We have: $
Y_{t+1}  = 
\begin{cases}
    Y_t + [\frac{Y_t+1}{2}]  \quad \quad &\text{with probability 18/37} \\
    Y_t - [\frac{Y_t+1}{2}]  &\text{Otherwise}
\end{cases}
$

Which yields, for $s \geq 1$:

\begin{align*}
    \E[Y_t-Y_{t+1}| Y_t = s] 
    &= s - \E[Y_{t+1}|Y_t = s] \\
    &= s - (\frac{18}{37}(s+[\frac{s+1}{2}])  
    + \frac{19}{37}(s-[\frac{s+1}{2}])) \\
    &= \frac{1}{37} [\frac{s+1}{2}] \\
\end{align*}

And hence: $\frac{s+1}{74} - \frac{1}{74}  < \E[Y_t - Y_{t+1}| Y_t = s] \leq \frac{s+1}{74}  \implies \frac{s}{74} < \E[Y_t - Y_{t+1}| Y_t = s] \leq \frac{s+1}{74} $ \\

As $h: s \mapsto \frac{s}{74}$ is monotonically increasing, using the Variable Drift Theorem would lead us to:

\[ \E[T | Y_0 = X_0 - 1 ] =    \]

\noindent
\textbf{(b)} 


\end{exo}

%exo1
\begin{exo}{\ \\}


\noindent
\textbf{(a)} The worst case is to have all pairs $\{a_i,a_j\}$ s.t: $i<j$ not ordered, in this case $INV(S_0) = \binom{n}{2} = \frac{n(n-1)}{2}$. We can achieve so, by taking a strictly ordered list, and inverse the order. \\

\noindent
\textbf{(b)} We let $S_t = [a_1, \dots, a_n]$ and $S_{t+1} = [b_1, \dots, b_n]$. We have, $b_i = a_j$ and $b_j = a_i$. We need to prove that:

\[ INV(S_{t+1}) \subsetneq  INV(S_{t}) \]

For the strictness, obsiously, $\{a_i, a_j\} \in INV(S_t)$ but $\{a_i, a_j\} = \{b_j, b_i\} \notin INV(S_{t+1})$, now we need to prove the inclusion: \\

Let $\{b_k, b_l\} \in INV(S_{t+1})$ s.t: $k < l$: we have ofc $b_l < b_k$ \\

\textbf{if} $k \neq i$ and $l \neq j$, then $(b_k, b_l) = (a_k, a_l)$ and hence, $\{a_k, a_l\} \in INV(S_t)$. \\

\textbf{If} $k=i$ and $l < j$:  \\

\textbf{If} $k=i$ and $i <  j < l$:  \\

\textbf{If} $k < i$ and $j = l$: \\
\textbf{If} $i<k<l$ and $j = l$: \\



\end{exo}

%exo1
\begin{exo}{\ \\}

\end{exo}

%exo3
\begin{exo}{\ \\}

Let $X_i$ be the RV associated with step $i$ of the algorithm and defined as follows:

$X_i = 
\begin{cases}
    1 \quad \quad \text{if at step $i$, we picked $j$ such as $B[j] = null$ (in one try)} \\
    0 \quad \quad \text{otherwise.} 
\end{cases}
$

 We have $\P[X_i = 1] = \frac{n-i+1}{n}$ \\




Now let $Y_i$ be the number of tries we need at each step $i$, to find a null element in $B$. Abviously, the running time $T$ could be expressed as follows: $T = \sum^{n}_{i=1} Y_i$, which yields, $\E[T] = \sum^{n}_{i=1} \E[Y_i]$ \\

$Y_i$ is a geometric RV, with parameter $p = \P[X_i = 1]$, hence, $\E[Y_i] = \frac{n}{n-i+1} $, which yields:

\begin{align*}
    \E[T] 
    &= \sum^{n}_{i=1} \frac{n}{n-i+1} \\
    &= n(1+ \frac{1}{2} + \dots + \frac{1}{n} ) \\
    &= n H_n \\
    &= \Theta(n\log(n))
\end{align*}






\end{exo}

%exo4
\begin{exo}{\ \\}

The case where $n=1$, is trivial, we will start our induction from $n=2$. \\


Let's say $A = [a_1, a_2]$, we have two permutations in total, $A_1 = A$ and $A_2 = [a_2, a_1]$. In this case, the main loop of the algorithm has two runs, but the second run (and generally when $i=n$) doesn't not change the outcome of the list. Hence, the list is only changed in the first step ($i = 1$). Which yields two possibilities with equal probability, we either pick $j=2$ (the algorithm swipes the two elements, and outputs, $A_2$), or we pick $j=1$, and nothing changes to the list, and the algorithm outputs $A_1$. \\


Now, we suppose that the algorithm is correct for some $n \geq 2$, and prove that the algorithm is correct for an input of size $n+1$: \\

Let, $T$ be a list of size $n+1$, and let $L =  [L_1,L']$ be a random permutation of $T$, where $L_1$ is the first element of $L$, and $L'$, is random permutaion of $T/\{L_1\}$ (i.e., the list $T$ minus the element $L_1$). We need to prove that $FastRandomPermutation(T)$, has a probability of $\frac{1}{(n+1)!}$ of outputing $L$.  \\

For the first step, the algorithm picks at uniformely random an element of T, and swipes it with the first elemenet of $T$ ($T_1$), Hence, we have get a probability of $\frac{1}{n+1}$  of actually picking $L_1$. For the rest, we are left with a list of size $n$, $T/\{L_1\}$, and since $L'$ is a valid permutation of $T/\{L_1\}$, we get by induction, that the probability pof picking $L'$ in the steps $i \in \{2,\dots , n+1\}$ is $\frac{1}{n!}$. \\

By the the multiplication theorem of probability, we conclude that the probability of picking $L$ as a permutaion is $\frac{1}{n+1} \frac{1}{n!} = \frac{1}{(n+1)!}$. This completes our proof by induction.
 
\end{exo}

\end{document}

