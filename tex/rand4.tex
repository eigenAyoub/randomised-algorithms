\documentclass{article}

\usepackage[utf8]{inputenc}
\usepackage[left=1.25in,top=1.25in,right=1.25in,bottom=1.25in,head=1.25in]{geometry}
\usepackage{amsmath,amssymb,amsthm}

\usepackage{import}
\usepackage{xifthen}
\usepackage{pdfpages}
\usepackage{transparent}

\newcommand{\incfig}[1]{%
    \def\svgwidth{\columnwidth}
    \import{./figures/}{#1.pdf_tex}
}



\newtheorem{exo}{Exercise}
\def\P{\mathbb{P}}
\def\E{\mathbb{E}}

\title{Randomised Algorithms \\
Winter term 2022/2023, Exercise Sheet No. 4}

\author{
    \textbf{Authors:} \\
    Ben Ayad, Mohamed Ayoub \\
    Kamzon, Noureddine
}

\begin{document}

\maketitle


\begin{exo}{\ \\}

\noindent
\textbf{(a)} Hey \\


\noindent
\textbf{(b)} Hey \\


\end{exo}

\begin{exo}{\ \\}

Let $C = \{x_1, \dots, x_N\}$ be a random cut of the graph, we are obviously interested in $\E[N]$, i.e., the expected number of edges in a cut.

Let $E = \{e_1, \dots, e_{|E|}\}$ and let the RV $X_i$ be the indicator of edge $e_i$ in $C$.


Clearly $N = \sum_{i}^{|E|} X_i$, and hence, $\E{N}= \sum_i^{|E|} \E X_i$

Now we prove that $\E (X_i) = 1/2$ 
\end{exo}



\end{document}

