\documentclass{article}

\usepackage[utf8]{inputenc}
\usepackage[left=1.25in,top=1.25in,right=1.25in,bottom=1.25in,head=1.25in]{geometry}
\usepackage{amsmath,amssymb,amsthm}

\usepackage{import}
\usepackage{xifthen}
\usepackage{pdfpages}
\usepackage{transparent}

\newcommand{\incfig}[1]{%
    \def\svgwidth{\columnwidth}
    \import{./figures/}{#1.pdf_tex}
}



\newtheorem{exo}{Exercise}
\def\P{\mathbb{P}}
\def\E{\mathbb{E}}

\title{Randomised Algorithms \\
Winter term 2022/2023, Exercise Sheet No. 4}

\author{
    \textbf{Authors:} \\
    Ben Ayad, Mohamed Ayoub \\
    Kamzon, Noureddine
}

\begin{document}

\maketitle


\begin{exo}{\ \\}

\noindent
\textbf{(a)} Hey \\


\noindent
\textbf{(b)} Hey \\


\end{exo}

\begin{exo}{\ \\}

Let $C = \{x_1, \dots, x_N\}$ be a random cut of the graph, where $\{x_i\}_{1\leq i \leq N}$ representes the edges. We are obviously interested in $\E[N]$, i.e., the expected number of edges in a cut. Let $E = \{e_1, \dots, e_{|E|}\}$ and let the RV $X_i$ be the indicator of edge $e_i$ in $C$.


Clearly $\displaystyle N = \sum_{i=1}^{|E|}X_i$, and hence, $\displaystyle \E[N]= \sum_i^{|E|} \E[X_i]$

Now we prove that $\E (X_i) = 1/2$. Suppose the edge $e_i$ connects the vertices $A$ and $B$. 

\begin{align*}
    \E[X_i] &= \P[X_i=1] \\
            &= \P[\{\text{A random cut contains $e_i$}\}] \\
            &= \P[\{\text{A cut contains one and only one of $A$ or $B$  }\}] \\
\end{align*}

Each cut is defined by a split of vertices $S_1 / S_2$, where $S_1$ selects $j \in \{1, \dots, |V|-1\}$ vertices at random from $V$. Each vertex has 1/2 probability to be in $S_1$ (resp. $S_2$). 
\begin{align*}
    \P[\{ (A,B)\in (S_1, S_2) \lor (A,B)\in (S_2, S_1)\}] &= \P[\{ (A,B)\in (S_1, S_2)\}] + \P[\{(A,B)\in (S_2, S_1)\}] \\
                                                          &= \P[\{ A \in S_1 \land B \in S_2 \}] + \P[\{  A \in S_1 \land B \in S_1\}] \\
                                                          &= \P[\{ A \in S_1\}]\P[\{ B \in S_2 \}] + \P[\{A \in S_1\}] \P[\{B \in S_1\}] \\
                                                          &= 1/2 1/2 + 1/2 1/2 = 1/2
\end{align*}


\end{exo}



\end{document}

