\documentclass{article}

\usepackage[utf8]{inputenc}
\usepackage[left=1.25in,top=1.25in,right=1.25in,bottom=1.25in,head=1.25in]{geometry}
\usepackage{amsmath,amssymb,amsthm}

\usepackage{import}
\usepackage{xifthen}
\usepackage{pdfpages}
\usepackage{transparent}

\newcommand{\incfig}[1]{%
    \def\svgwidth{\columnwidth}
    \import{./figures/}{#1.pdf_tex}
}



\newtheorem{exo}{Exercise}
\def\P{\mathbb{P}}
\def\E{\mathbb{E}}

\title{Randomised Algorithms \\
Winter term 2022/2023, Exercise Sheet No. 5}

\author{
    \textbf{Authors:} \\
    Ben Ayad, Mohamed Ayoub \\
    Kamzon, Noureddine
}

\begin{document}

\maketitle

%exo1
\begin{exo}{\ \\}
\noindent
\textbf{(a)} We have established  in the notes that the probability of not making any mistakes after $n-t$ contractions is at least $\frac{t(t-1)}{n(n-1)}$, in this case, when $t = n-1$, we would get that this probability is $\frac{n-2}{n} = 1 - \frac{2}{n} $, which is in fact equal to $q_n$. \\

\noindent
\textbf{(b)} We denote the probability that $S_1$ (resp $S_2$) is correct as $P_1$ (resp. $P_2$), we have: 
\begin{align*}
P_1 &= \P[\text{\{Output of L5 is correct\}}|\text{\{Contraction at L4 is correct\}}]\P[\text{\{Contraction at L4 is correct\}}] \\
    &= P(n-1) q_n \\
P_2 &= P(n)
\end{align*}

The algorithm is succesful if either the return at line 7 was a succes (i.e., $S_1$ is a success) or  the return at line 10 was a succes, which yiled the following: 

\begin{align*}
    p(n) &= \P[\text{\{Entered L7\}}]\P[\text{\{Succes\}}|\text{\{entered L7\}}]
    + \P[\text{\{Didn't Enter L7\}}]\P[\text{\{Succes\}}|\text{\{Didnt enter L7\}}] \\
         &= q_n \P[\text{\{Succes\}}|\text{\{entered L7\}}]
    + (1-q_n)\P[\text{\{Succes\}}|\text{\{Didnt enter L7\}}] \\
         &= q_n \P[\text{\{$S_1$ is succesful\}}]
    + (1-q_n)\P[\text{\{The best of $S_1$, $S_2$ was succesful\}}] \\
         &= q_n \P[\text{\{$S_1$ is succesful\}}]
    + (1-q_n)(1 - \P[\text{\{$S_1$ and $S_2$ failed\}}]) \\
         &= q_n P_1 + (1-q_n)[1- (1-P_1)(1-P_2)]  \quad \quad \text{(We will just keep re-arranging after now)} \\
         &= q_n^2 P(n-1) + (1-q_n)[1 - (1-q_nP(n-1))(1-P(n))] \\
         &= q_n^2 P(n-1) + (1-q_n)[P(n) + q_nP(n-1) - q_nP(n-1)P(n)] \\ 
         &= q_nP(n-1) + (1-q_n)P(n) - q_n(1-q_n)P(n-1)P(n) \\ 
    \implies & \\
         q_nP(n)&= q_nP(n-1) - q_n(1-q_n)P(n-1)P(n) \\ 
    \implies & \\
         P(n) &= P(n-1) - (1-q_n)P(n-1)P(n) \quad \quad (q_n \neq 0 \text{ since $n\neq 2$ as we would verify line 1 if $n=2$} )
\end{align*}

\noindent
\textbf{(c)} By dividing the equation that we derived in the last question by $P(n)P(n-1)$, we get the following, for $k \in \{3, \dots, n\}$:

\begin{align*}
    \frac{1}{P(k-1)} - \frac{1}{P(k)}  &= - \frac{2}{k}  \\
    \implies & \\
    \sum_{k=3}^{n} \frac{1}{P(k-1)} - \sum_{k=3}^{n} \frac{1}{P(k)}  &=
    - \sum_{k=3}^{n} \frac{2}{k}  \\
    \frac{1}{P(2)} - \frac{1}{P(n)}  &= -\sum_{k=3}^{n} \frac{2}{k}  \\
    \implies &\\
    P(n) = \frac{1}{\sum_{k=3}^{n} \frac{2}{k} -1} 
\end{align*}

Comparisons of the probability of succes with FastCut:

\begin{itemize} 
    \item GeoContraction has a $\Theta(\frac{1}{\log(n)})$ (Assuming the bouds are exact like suggested in \textbf{(b)} )
    \item FastCut had a $\Omega(\frac{1}{\log(n)})$ 
\end{itemize}


\end{exo}

%exo2
\begin{exo}{\ \\}
\noindent
\textbf{(a)}

\noindent
\textbf{(b)} 

\noindent
\textbf{(c)}  
\end{exo}

%exo3
\begin{exo}{\ \\}
\noindent
\textbf{(a)}

\noindent
\textbf{(b)} 

\end{exo}


%exo4
\begin{exo}{\ \\}
\noindent
\textbf{(a)}

\noindent
\textbf{(b)} 

\noindent
\textbf{(c)}  

\noindent
\textbf{(d)} 
\end{exo}


\end{document}

