\documentclass{article}

\usepackage[utf8]{inputenc}
\usepackage[left=1.25in,top=1.25in,right=1.25in,bottom=1.25in,head=1.25in]{geometry}
\usepackage{amsmath,amssymb,amsthm}

\usepackage{import}
\usepackage{xifthen}
\usepackage{pdfpages}
\usepackage{transparent}

\usepackage{algpseudocode}

\newcommand{\incfig}[1]{%
    \def\svgwidth{\columnwidth}
    \import{./figures/}{#1.pdf_tex}
}



\newtheorem{exo}{Exercise}
\def\P{\mathbb{P}}
\def\E{\mathbb{E}}

\title{Randomised Algorithms \\
Winter term 2022/2023, Exercise Sheet No. 8}

\author{
    \textbf{Authors:} \\
    Ben Ayad, Mohamed Ayoub \\
    Kamzon, Noureddine
}

\begin{document}

\maketitle

\begin{exo}{\ \\}
\noindent
\textbf{(a)}  The algorithm described briefly as follows, has a \textbf{cubic} runtime.

\begin{algorithmic}[1]
\For{$1 \leq i \leq n, 1 \leq j \leq n$} 
    \State $\alpha_{i,j} \gets sum([A_{i,k}B_{k,j} \text{ : } k \in \{1,\dots ,n\}])$ 
    \State $\alpha_{i,j} \gets \alpha_{i,j} - c_{i,j}$
\If{$\alpha_{i,j} \neq 0$}
    \State \Return $0$
\EndIf
\EndFor 
\State \Return $1$
\end{algorithmic}


\noindent
\\\textbf{(b)}  Computing $x$, $y$, and $z$ requires $\mathcal{O}(n^2)$ each, and computing $t = y-z$ requires $\mathcal{O}(n)$, hence, the asymptotic runtime of this RA is $\mathcal{O}(n^2)$.\\

\noindent
\textbf{(c)} If $r \neq 0$, the event $\mathcal{E}$ implies the following:
\begin{itemize}
    \item $r \in Ker(D)$
    \item $r$ is orthogonal to all rows of $D$
    \item Since $D \neq 0$, there exists at least one row $d_i \neq 0$ s.t: $d_i^\top r = 0$
    \item $\sum^{n}_{i=1} r_i D_{.,i} = 0$
\end{itemize}

We have:

\begin{align*}
    \P[\mathcal{E}] 
    &= \P[Dr=0|r=0]\P[r=0] + \P[Dr=0|r\neq0]\P[r\neq0] \\
    &= \frac{1}{3^n} + \P[Dr=0|r\neq0](1-\frac{1}{3^n}) \quad \quad  \text{\textbf{[1]}}
\end{align*}


For some $k\leq n$, we let, $d_1, \dots, d_k$  be the rows of $D$ that are not equal to $0$. We have:
\begin{align*}
    \P[Dr=0|r\neq0]
    &= \P[d_1^\top r = 0, \dots, d_k^\top r = 0 |  r\neq0] \\
    &\leq \P[d_1^\top r = 0 | r\neq0]    \quad \quad  \text{\textbf{[2]}}
\end{align*}

Let $d_{1,j}$ be the last element of $d_1$ not equal to $0$, we have: $d_1^\top r=0$ \textbf{if and only if} after $j-1$ picks of $r_1, \dots , r_{j-1}$, the $j^{th}$ pick (i.e., $r_j$) is chosen s.t:   $-d_{1,j} r_j = \sum^{j-1}_{i=1} d_{1,i} r_i $. Which yields the following:


\begin{align*}
    \P[d_1^\top r = 0 | r\neq0]
    &= \P\left[r_j = \sum^{j-1}_{i=1} \frac{-d_{1,i}}{d_{1,j}} r_i |r\neq0\right] \\
    &= \frac{\P\left[r_j = \sum^{j-1}_{i=1} \frac{-d_{1,i}}{d_{1,j}} r_i , r\neq0\right]}{\P[r\neq 0]} 
    \quad \quad \text{(We refer to $r' = [r_1, r_2, \dots, r_{j-1}]$)}
    \\ 
    &= \frac{
        \P\left[r_j = \sum^{j-1}_{i=1} \frac{-d_{1,i}}{d_{1,j}} r_i , r\neq0\right, r' \neq 0]
            \P[r' \neq 0]+
        \P\left[r_j = \sum^{j-1}_{i=1} \frac{-d_{1,i}}{d_{1,j}} r_i , r\neq0\right, r' = 0 ]
            \P[r' = 0]
    }
    {\P[r\neq 0]} \\ 
\end{align*}

For the second term, we have: ($r' = 0 \implies r_j =0$) and ($r' = 0$ and $r \neq 0 \implies r_j \neq 0$), hence the second term is equal to $0$. Which yields:

\begin{align*}
    \P[d_1^\top r = 0 | r\neq0]
    &\leq 
        \P\left[r_j = \sum^{j-1}_{i=1} \frac{-d_{1,i}}{d_{1,j}} r_i , r\neq0, r' \neq 0\right]
         \frac{\P[r' \neq 0]}{\P[r\neq 0]} \\
    &\leq \frac{1}{3} \frac{\P[r' \neq 0]}{\P[r\neq 0]} \\
    &= \frac{1}{3}  \frac{3^{n-1}-1}{3^{n-1}}  \frac{3^n}{3^n-1} \\
    &= \frac{3^{n-1}-1}{3^n - 1} 
\end{align*}

Using \textbf{[2]}, we get $\P[D r = 0 | r\neq0] \leq  \frac{3^{n-1}-1}{3^n - 1}$, and from \textbf{[1]}, we conclude that:

\[
\P[\mathcal{E}] \leq \frac{1}{3^n} +  \frac{3^{n-1}-1}{3^n - 1}(1- \frac{1}{3^n} ) = \frac{1}{3} 
\]

\noindent
\textbf{(d)} We can reduce the probability by amplification.

\end{exo}

\begin{exo}{\ \\}


\noindent 
\textbf{(a)} We condider the following Algorithm $T_k$, we choose $k = [\sqrt{2n]} + 1$. \\



\begin{algorithmic}[1]
\State \text{We generate $k$ images, as follows: $e_1 = h(1), \dots, e_k = h(k)$}
\If{A collision happens} 
    \State \Return $s=n$
\EndIf
\State \Return $s = n^2$
\end{algorithmic}

\noindent
\\\textbf{(b)} We will refer to the probability of $T_k$ being correct as $\P[T]$, we have: 

\begin{align*}
    \P[T] 
    &= \P[s=n]\P[T|s=n] + \P[s=n^2]\P[T|s=n^2] \\
    &= \frac{1}{2} (\P[T|s=n] + \P[T|s=n^2]) \\
    &= \frac{1}{2} (\P[\text{A collision happening in $T_k$}|s=n] 
    + \P[\text{No collisions happening in $T_k$}|s=n^2]) \\
\end{align*}

From the lecture notes, we have, $\P[\text{A collision happening in $T_k$}|s=n] > \frac{1}{2}$, also:

\begin{align*}
    \P[\text{No collisions happening in $T_k$}|s=n^2]) 
    &= \prod_{i=2}^{k} (1-\frac{i-1}{n^2})  \\
    &\geq (1-\frac{k-1}{n^2})^k 
    \quad \quad \text{(using the Bernoulli Inequality)}\\
    &\geq 1 - \frac{k(k-1)}{n^2} \\
    &> \frac{1}{2}
    \quad \quad \text{(as $\frac{n}{2} > k)$}\\
\end{align*}

Which finally yields:
\begin{align*}
    \P[T] 
    &= \frac{1}{2} (\P[\text{A collision happening in $T_k$}|s=n] 
    + \P[\text{No collisions happening in $T_k$}|s=n^2]) \\
    &> \frac{1}{2}(\frac{1}{2} + \frac{1}{2} ) \\
    &= \frac{1}{2}
\end{align*}

\newpage
\end{exo}

\begin{exo}{\ \\}
\noindent    
\textbf{(a)} We have:

\begin{align*}
    \sum_{i\geq1} \P[X \geq i]
    &= \sum_{i\geq1} \sum_{j = i }^{\infty} \P[X=j] \\
    &= \sum_{j\geq1} \sum_{i=1}^{j} \P[X=j] \\
    &= \sum_{j\geq1} \P[X=j]\sum_{i=1}^{j}1  \\
    &= \sum_{j\geq1} j \P[X=j] = \E[X]\\
\end{align*}

If $X$ can only take negative values, then: 
\begin{align*}
    \E[X] 
    &= -\E[-X] \\
    &= -\sum_{j \geq 1} \P[-X \geq j] \\
    &= -\sum_{j \geq 1} \P[X \leq -j]
\end{align*}



\noindent    
\textbf{(b)} Let $\P[X > 0] = p_X$, and $\P[Y > 0] = p_Y$, we have: $p_X \leq p_Y$, we suppose that $0 < p_X < 1$ and $ 0 <  p_Y < 1$:

\begin{align*}
    \E[Y]-\E[X] 
    &= p_Y \E[Y|Y>0] + (1-p_Y)\E[Y|Y<0] -
    p_X \E[X|X>0] - (1-p_X)\E[X|X<0] \\
    &= p_Y \E[Y|Y>0] - p_X \E[X|X>0] 
    +(1-p_Y)\E[Y|Y<0] -  (1-p_X)\E[X|X<0] 
    \quad \quad \text{\textbf{Eq. $\#42$}}
\end{align*}

We have, for $i \geq 1$:

\begin{align*}
    \P[X\geq i] \leq \P[Y \geq 1] 
    &\implies \sum_{k \geq 1} \P[X\geq k] \leq \sum_{k \geq 1} \P[Y \geq k]\\
    &\implies p_X \sum_{k \geq 1}  \frac{\P[X\geq k]}{p_X}
    \leq  p_Y \sum_{k \geq 1} \frac{\P[Y \geq k]}{p_Y} \\
    &\implies  p_X \E[X | X>0] \leq p_Y \E[Y | Y>0]
    \quad \quad \text{(using the result of the past question)}  \\
    &\implies p_Y \E[Y | Y>0] - p_X \E[X | X>0] \geq 0 \quad \quad  \text{\textbf{[1]}}
\end{align*}

On the other side, we have:
\[
    \E[X|X<0] = -\sum_{i \geq 1} \frac{\P[X \leq -i]}{1-p_X} \implies 
    (1-p_X) \E[X|X<0] = -\sum_{i \geq 1} \P[X \leq -i]
\]

Similarily:

\[
    (1-p_Y) \E[Y|Y<0] = -\sum_{i \geq 1} \P[Y \leq -i]
\]

Hence,

\begin{align*}
     (1-p_Y) \E[Y|Y<0] -  (1-p_X) \E[X|X<0]
     &= \sum_{i \geq 1} \P[X \leq -i] - P[Y \leq -i] \\
     &= \sum_{i \geq 1} \P[Y \geq i] - P[X \geq i] \\
     &\geq 0   \quad \quad \text{\textbf{[2]} }
\end{align*}

By using the results in \textbf{[1]} and \textbf{[2]} in \textbf{Eq. $\#42$}, we conclude that $\E[X] \leq \E[Y]$.
\end{exo}



\end{document}
